\documentclass[french]{beamer}

\usepackage[utf8]{inputenc}
\usepackage[T1]{fontenc}
\usepackage{lmodern}
\usepackage{amsmath, amssymb}
\usepackage{verbatim}

\usepackage{babel}
\usepackage{hyperref} % permet d'obtenir des liens sur les sections dans les PDF
\newcommand\justifyit{\rightskip0pt \leftskip0pt}


%CHOIX DU THEME et/ou DE SA COULEUR
% => essayer différents thèmes (en décommantant une des trois lignes suivantes)
%\usetheme{PaloAlto}
\usetheme{Madrid}
%\usetheme{Copenhagen}

% => il est possible, pour un thème donné, de modifier seulement la couleur
%\usecolortheme{crane}
%\usecolortheme{seahorse}

\useoutertheme[left]{sidebar}
%
%\usepackage[backend=biber]{biblatex} 
%\addbibresource{main.bib}

%%%%%%%%%%%%%%%%%%%%%%%%%%%%%%%%%%%%

%%%%%%%%%%%%%%%%%%%%%%%%%%%%%%%%%%

\setbeamertemplate{bibliography item}[triangle]



\def\uC{$\mu$C}
%Pour le TITLEPAGE
\title{Interruption Timer0 avec 18F4550}
\subtitle{https://github.com/aymentabib/PicMicroEx}
%\author[Nom (court)]{Nom (long) de l'auteur}
\author[]{Aymen Tabib}
\date{Février 2017}
\institute[DIY]{https://diyelectronique.wordpress.com/}
%


\begin{document}

\begin{frame}
	\titlepage
\end{frame}

%\begin{frame}
%	Un environnement \texttt{frame} pour chaque \emph{diapositive}.
%	\visible<2>{Chaque diapo pouvant contenir plusieurs \emph{couches}.}
%\end{frame}


\begin{frame}{Sommaire}
	\tableofcontents
\end{frame}

\section{Introduction}
\begin{frame}{Introduction}
\justifyit
Dans cette video on va voir:
\begin{itemize}
\item comment configurer le TIMER0 du PIC18F4550.
\item comment activer l'interruption Timer0;
\item calculer le temps entre deux interruptions successives.
\item simulation du code sur isis.
\end{itemize}


\end{frame}


\section{Le Timer c'est quoi ?}

\begin{frame}{Le Timer c'est quoi ?}
\justifyit
Le timer0 est  un compteur. Mais qu'allez-vous compter avec ce
timer? Et bien,pour le Timer0 du PIC vous avez deux possibilités :

\begin{itemize}

\item Vous pouvez compter les impulsions reçues sur la pin RA4/TOKI.
Nous dirons dans ce cas que nous sommes en mode compteur.
\item Vous pouvez aussi décider de compter les cycles d’horloge du PIC® lui-même.
Dans ce cas, comme l'horloge est fixe, nous compterons donc en réalité du temps.
Donc, nous serons en mode timer.
\end{itemize}
\end{frame}


\begin{frame}{Le Timer c'est quoi ?}
\justifyit
dans le datasheet on trouve ça dans les lignes suivantes:
\begin{itemize}
\item Selectable clock source (internal or external). (page 125)
\item In Timer mode, the module increments on every clock... (p126)
\item In Counter mode, Timer0 increments either on every rising or falling edge of pin RA4/T0CKI.(p126)
\end{itemize}

\end{frame}

\section{Configuration du Timer0}
\begin{frame}[fragile]{configuration du prédiviseur(prescaler)}
\justifyit
Comme on voit sur la figure 11-1 page 126 du DS(datasheet, il y a prédiviseur avant le 
Timer0, ce qui nous permet de déviser la vitesse d'incrémentation du timer.Pour le mode timer
la fréquence d'incrémentation Fcyc= Fosc/4 on peut la faire diviser par 2,4,..,128 ou 256.
les bits qui configurent le prescaler sont T0PS2:T0PS0 qui sont dans le resgistre T0CON.
dans notre cas on va mettre le prescaler à une division par 4:
\begin{verbatim}

T0PS0 = 1;
T0PS1 = 0;
T0PS2 = 0;
\end{verbatim}

\end{frame}

\begin{frame}[fragile]{Activation de prescaler}
\justifyit
Sur la figure 11-1 on va aussi que le signal d'incrémentation du timer peut passer directemnt
au timer sans etre diviser par le presclare. Donc il faut activer le chemin qui passe par le presclare avec le bit :
\begin{verbatim}
/*
    PSA: Timer0 Prescaler Assignment bit
    1 = TImer0 prescaler is NOT assigned. 
    		Timer0 clock input bypasses prescaler.
    0 = Timer0 prescaler is assigned. 
    		Timer0 clock input comes from prescaler output.
*/
PSA = 0;
\end{verbatim}

\end{frame}

\begin{frame}[fragile]{Sélection de la source d'horloge}
\justifyit
Comme on a vu précédemment le TIMER0 peut etre incrémenter par l'horloge interne du pic ou par une sorce externele bit qui selectionne cette source est:
\begin{verbatim}
  /*
    T0CS: Timer0 Clock Source Select bit
    1 = Transition on T0CKI pin
    0 = Internal instruction cycle clock (CLKO)
    */
    T0CS = 0;
\end{verbatim}



\end{frame}



\begin{frame}[fragile]{8-bit/16-bit}
\justifyit
Le Timer0 du 18F4550 peut fonctionner comme un compteur 16bits ou 8bits seulement.
Pour sélectionner le mode de fonctionnement:
\begin{verbatim}
/*
    T08BIT: Timer0 8-Bit/16-Bit Control bit
    1 = Timer0 is configured as an 8-bit timer/counter
    0 = Timer0 is configured as a 16-bit timer/counter
 */
T08BIT = 0;
\end{verbatim}


\end{frame}




\begin{frame}[fragile]{Mettre en marche le Timer0}
\justifyit
Lors de la mise sous tension le Timer0 ne compte pas et vous avez le choix de l'activer et de les désactiver quand vous voulez:
\begin{verbatim}
/*
    TMR0ON: Timer0 On/Off Control bit
    1 = Enables Timer0
    0 = Stops Timer0
*/
TMR0ON = 1;
\end{verbatim}
\end{frame}

\section{Configuration de l'interruption du Timer0}
\begin{frame}[fragile]{Activation/Désactivation de la priorité}
\justifyit
Pour les Pic 18F vous pouvez donner une priorité haute ou basse pour les interruptions
que vous utilisé, mais vous pouvez aussi désactivé cette notion de priorité pour laissé votre code compatible avec d'autre Pic qui n'ont pas cette option:
\begin{verbatim}
//IPEN:
//0: pas de priorité entre les inter,
//1: il ya priorité H et L
IPEN = 0;
\end{verbatim}

\end{frame}

\begin{frame}[fragile]{Activation de l'interruption}
\justifyit
l'activation de l'interruption Timer0 nécessite la modification de deux bits:
\begin{verbatim}
/* activer les interruptions
si ce bit est à 0, aucune interruption ne sera 
déclaré même si son bit Enable est mit à 1.
c'est le "disjoncteur" principal des interruptions
*/ 
INTCONbits.GIE = 1;
// activer l'interruption timer 0
TMR0IE = 1;
\end{verbatim}

\end{frame}

\begin{frame}[fragile]{CALCUL :)}
\justifyit
Le but d'avoir une interruption avec un timer est d'effectuer une tache bien déterminer à un instant bien précis.
Dans notre cas, envoyer une impulsion au servomoteur chaque 20ms.
notation:
\begin{verbatim}
Fosc : frequnce clock cpu dans notre cas 48Mhz
Fcyc : Fosc/4
prescaler :la valeur du prescaler du timer 1.
OF :(overFlow)le nombre d'incrémentation pour
    atteindre 0xffff, par defaut c'est (0xffff+1)
t_int:la durée entre deux interruptions successives
\end{verbatim}
on peut le changer OF en ecrivant une valeur offset dans le timer dès qui 'il passe de 0xffff
à 0x0000, dans ce cas: 
    \begin{verbatim}
    OF = (0xffff+1)-offset 
    offset = (0xffff+1)-OF
    \end{verbatim}
 
\end{frame}

\begin{frame}[fragile]{suite calcul}
\justifyit
   \begin{verbatim}
t_int = (1/Fosc) * 4 *OF*prescaler
t_int  = (1/(48*10^6))*4*OF*prescaler =20ms
OF = 20ms/((1/(48*10^6))*4*prescaler)
offset = (0xffff+1)-20ms/((1/(48*10^6))*4*prescaler)
pour un prescaler de 4.
offset = 5536
    \end{verbatim}
    Pour avoir une interruption chaque 20ms on doit écrire le valeur 5536 dans le timer dès qu'il passe à zero.
    càd au début de l'interruption.

\end{frame}

\begin{frame}[fragile]{}


\raggedright
    \nocite{*}
    \bibliographystyle{plain}
    \bibliography{main}
%\addcontentsline{toc}{chapter}{Bibliographie}


\end{frame}


%\begin{frame}[fragile]{}
%\justifyit
%\end{frame}

\end{document}